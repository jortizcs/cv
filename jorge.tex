% resume.tex
%
% (c) 2002 Matthew Boedicker <mboedick@mboedick.org> (original author) http://mboedick.org
% (c) 2003 David J. Grant <dgrant@ieee.org> http://www.davidgrant.ca
% (c) 2007 Todd C. Miller <Todd.Miller@courtesan.com> http://www.courtesan.com/todd
% (c) 2009-2012 Derek R. Hildreth <derek@derekhildreth.com> http://www.derekhildreth.com 
%This work is licensed under the Creative Commons Attribution-NonCommercial-ShareAlike License. To view a copy of this license, visit http://creativecommons.org/licenses/by-nc-sa/1.0/ or send a letter to Creative Commons, 559 Nathan Abbott Way, Stanford, California 94305, USA.

% GENERAL NOTE:  There may be some notes specific to myself.  If you're only interested in my LaTeX source or it doesn't make sense, please disregard it.

\documentclass[letterpaper,11pt]{article}

%-----------------------------------------------------------
\usepackage{latexsym}
\usepackage[empty]{fullpage}
\usepackage[usenames,dvipsnames]{color}
\usepackage{verbatim}
\usepackage[pdftex]{hyperref}
\usepackage[backend=bibtex,maxnames=99]{biblatex}
\usepackage[usenames,dvipsnames]{xcolor}
\addbibresource{pubs.bib}
\hypersetup{
    colorlinks,%
    citecolor=black,%
    filecolor=black,%
    linkcolor=black,%
    urlcolor=black 
    %urlcolor=mygreylink     % can put red here to better visualize the links
}
\urlstyle{same}
\definecolor{mygrey}{gray}{.85}
\definecolor{mygreylink}{gray}{.40}
\textheight=9.0in
\raggedbottom
\raggedright
\setlength{\tabcolsep}{0in}

% Adjust margins
\addtolength{\oddsidemargin}{-0.375in}
\addtolength{\evensidemargin}{0.375in}
\addtolength{\textwidth}{0.5in}
\addtolength{\topmargin}{-.375in}
\addtolength{\textheight}{0.75in}

%-----------------------------------------------------------
%Custom commands
\newcommand{\resitem}[1]{\item #1 \vspace{-2pt}}
\newcommand{\resheading}[1]{{\large \colorbox{mygrey}{\begin{minipage}{\textwidth}{\textbf{#1 \vphantom{p\^{E}}}}\end{minipage}}}}
\newcommand{\ressubheading}[4]{
\begin{tabular*}{6.5in}{l@{\extracolsep{\fill}}r}
		\textbf{#1} & #2 \\
		\textit{#3} & \textit{#4} \\
\end{tabular*}\vspace{-6pt}}

\newcommand{\ressubsubheading}[2]{
\begin{tabular*}{6.5in}{l@{\extracolsep{\fill}}r}
		\textit{#1} & \textit{#2} \\
\end{tabular*}\vspace{-6pt}}
%-----------------------------------------------------------

%-----------------------------------------------------------
%General Resume Tips
%   No periods!  Technically, nothing in this document is a full sentence.
%   Use parallelism by ending key words with the same thing,  i.e. "Coordinated; Designed; Communicated".
%   More tips on bottom of this LaTeX document.
%-----------------------------------------------------------



\begin{document}



\newcommand{\mywebheader}{
\begin{tabular*}{7in}{l@{\extracolsep{\fill}}r}
	\textbf{\href{http://researcher.ibm.com/researcher/view.php?person=us-jjortiz}{\LARGE Jorge Ortiz, Ph.D.}} & \href{mailto:jortiz@alum.mit.edu}{jortiz@alum.mit.edu}\\
	{\footnotesize \texttt{\colorbox{gray}{[601 West 57 Street, Apt 5Q, NY, NY 10019]}}} & \href{http://researcher.ibm.com/researcher/view.php?person=us-jjortiz}{http://jorgeortizphd.info} \\
	\end{tabular*}
\\
\vspace{0.1in}}

% CHANGE HEADER SOURCE HERE
\mywebheader

%%%%%%%%%%%%%%%%%%%%%%
\resheading{Education}
	\begin{itemize}
        \item
            \ressubheading{\href{}{University of California, Berkeley}}{Berkeley, CA}{\href{}{Doctor of Philosophy in Computer Science}; \href{}}{May. 2010 -- Dec. 2013}
                { \footnotesize
				\begin{itemize}
					\resitem{Dissertation: A Platform Architecture for Sensor Data Processing and Verification in Buildings}
				\end{itemize}
				}
        \item
            \ressubheading{\href{}{University of California, Berkeley}}{Berkeley, CA}{\href{}{Masters of Science in Computer Science}; \href{}}{Aug. 2007 -- May 2010}
                { \footnotesize
				\begin{itemize}
					\resitem{Thesis: Multichannel Reliability Assessment in Real World WSNs}
                    \resitem{Relevant Courses: Advanced Systems Seminar I \& II, Graduate Networking, Sensor Networks Seminar, Combinatorial Algorithms, Parallel Computation Algorithms, Statistical Learning Theory, Practical Machine Learning.}
				\end{itemize}
				}
       \item
            \ressubheading{\href{}{Massachusetts Institute of Technology}}{Cambridge, MA}{\href{}{Bachelors of Science in Computer Science and Engineering}; \href{}}{Aug. 1999 -- May 2003}
                { \footnotesize
				\begin{itemize}
					\resitem{Thesis: Connection Oriented Routing Environment (CORE): A Generalized Device         Interconnect}
				\end{itemize}
				}
    \end{itemize} % End Education list

%%%%%%%%%%%%%%%%%%%%%%
\resheading{Experience}
	\begin{itemize}

        \item 
			\ressubheading{\href{http://www.research.ibm.com/}{IBM Research}}{Yorktown Heights, NY}
				{Research Staff Member}{Dec. 2013 -- Present}
				{ \footnotesize
				\begin{itemize}
                    \resitem{Research work on a variety of topics related to distributed systems, mobile sensing, cloud-based middleware architectures, and machine learning. }
				\end{itemize}
				}


        \item 
			\ressubheading{\href{https://spire.com/}{Spire}}{San Francisco, CA}
				{Senior Software Engineer}{Jan. 2013 -- Sept. 2013}
				{ \footnotesize
				\begin{itemize}
                    \resitem{Designed and wrote communication kernel for arduino-based, nano satellites.}
				\end{itemize}
				}

		\item 
			\ressubheading{\href{http://www.oracle.com}{Oracle Corporation}}{Burlington, MA}
				{Software Engineer}{Sept. 2003 -- Feb. 2007}
				{ \footnotesize
				\begin{itemize}
                    \resitem{Assisted in designing, debugging, and maintaining several features in Oracle Enterprise Planning and Budgeting (EPB) software suite; include, but not limited to the setup of PL/SQL packages, schemas, and interfaces to the Java\-based UI.}
				\end{itemize}
				}

				
		
\end{itemize}  % End Experience list

%%%%%%%%%%%%%%%%%%%%%%
\resheading{\href{}{Skills}}
	\begin{description}
		\item[Proficient in:]{Java, Python, C
		}
		\item[Experience/Familiar with:]{ 
            C++, Javascript, Perl, NesC, TinyOS, Android development
		}
        \item[Fluent in:] { 
            Spanish
		}
	\end{description} % End Skills list


\resheading{Projects}
\begin{itemize}
    \item {\textbf{Systems for Large-Scale Distributed Machine Learning}}
            \begin{itemize}
                \resitem{Application programmers in domains like machine learning, scientific computing, and computational biology are accustomed to using powerful, high productivity array languages such as MatLab, R and NumPy. Distributed array frameworks aim to scale array programs across machines. High performance is necessary to facilitate broad experimentation.  We have been examining various fundamental algorithms and system designed for increasing machine leanring job performance and completetion time.}
                \resitem{Spartan~\cite{spartan} maximizes locality of access to distributed arrays. Such locality is critical for high performance.  Spartan is a distributed array framework that automatically determines how to best partition (aka “tile”) n-dimensional arrays and to co-locate data with computation to maximize locality.}
            \end{itemize}

    \item {\textbf{Systems for Distributed Machine Learning in Resource-constrained Environments}}
        \begin{itemize}
            \resitem{Machine learning system for running machine learning in resource constrained environments, such as mobile phones.}
            \resitem{Approximation techniques for cooperatively clustering images taken by phones in the same location, in real time.}
            \resitem{Design of mechanisms and APIs for separating machine learning algorithm design from the distribued system complexities for lossy links and high-failure rates of mobile environments} 
        \end{itemize}

    \item {\textbf{Automatic Metadata Normalization of Sensor Feeds}}
        \begin{itemize}
            \resitem{Machine learning techniques for characterizing timeseries sensor data into class types.}
            \resitem{Active learning techniques where string formats are used to automatically parse contextual metadata embedded in sensor tag names~\cite{Arka_buildsys2015}}.
            \resitem{Transfer learning techniques to use models across sensor deployments in the similar deployment domains.  For example, metadata models that can be applied across buildings~\cite{Fontugne:2013:SBS:2461381.2461399,Hong:2013:TAS:2528282.2528302,Dezhi_buildsys2015}}.
        \end{itemize}
\end{itemize}

%\resheading{Internships}
%	\begin{itemize}
%
%        \item 
%			\ressubheading{\href{}{Charles River Analytics}}{Cambridge, MA}
%				{Research Intern}{Jun. 2003 -- Sept. 2003}
%				{ \footnotesize
%				\begin{itemize}
%                    \resitem{Designed genetic algorithms for optimizing flight paths under various weather conditions.  Tested algorithms in a cluster.  Wrote cluster management components of simulator.}
%				\end{itemize}
%				}
%
%
%        \item 
%			\ressubheading{\href{}{IBM Research}}{Cambridge, MA}
%				{Research Intern}{May 2002 -- Aug. 2002}
%				{ \footnotesize
%				\begin{itemize}
%                    \resitem{Designed backend of prototype mobile email client.  Designed usability experiments and collected data about fundamental components in the design.}
%				\end{itemize}
%				}
%
%        \item 
%			\ressubheading{\href{}{Merrill Lynch}}{New York, NY}
%				{Summer Analyst}{May -- Aug. 2000/2001}
%				{ \footnotesize
%				\begin{itemize}
%                    \resitem{Design and wrote internal websites for Technology Infrastructure Services- Network Services Group (TIS-NS) the Corporate and Institutional Client Group Desktop Engineering (CDE).}
%				\end{itemize}
%				}
%
%	\end{itemize}  % End Experience list



%%%%%%%%%%%%%%%%%%%%%%

\resheading{Select Publications}
\nocite{*}                                          % Print all publications.
\printbibliography[heading=none]



\end{document}



